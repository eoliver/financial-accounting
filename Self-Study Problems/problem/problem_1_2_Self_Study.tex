\documentclass{article}\usepackage[]{graphicx}\usepackage[]{xcolor}
% maxwidth is the original width if it is less than linewidth
% otherwise use linewidth (to make sure the graphics do not exceed the margin)
\makeatletter
\def\maxwidth{ %
  \ifdim\Gin@nat@width>\linewidth
    \linewidth
  \else
    \Gin@nat@width
  \fi
}
\makeatother

\definecolor{fgcolor}{rgb}{0.345, 0.345, 0.345}
\newcommand{\hlnum}[1]{\textcolor[rgb]{0.686,0.059,0.569}{#1}}%
\newcommand{\hlstr}[1]{\textcolor[rgb]{0.192,0.494,0.8}{#1}}%
\newcommand{\hlcom}[1]{\textcolor[rgb]{0.678,0.584,0.686}{\textit{#1}}}%
\newcommand{\hlopt}[1]{\textcolor[rgb]{0,0,0}{#1}}%
\newcommand{\hlstd}[1]{\textcolor[rgb]{0.345,0.345,0.345}{#1}}%
\newcommand{\hlkwa}[1]{\textcolor[rgb]{0.161,0.373,0.58}{\textbf{#1}}}%
\newcommand{\hlkwb}[1]{\textcolor[rgb]{0.69,0.353,0.396}{#1}}%
\newcommand{\hlkwc}[1]{\textcolor[rgb]{0.333,0.667,0.333}{#1}}%
\newcommand{\hlkwd}[1]{\textcolor[rgb]{0.737,0.353,0.396}{\textbf{#1}}}%
\let\hlipl\hlkwb

\usepackage{framed}
\makeatletter
\newenvironment{kframe}{%
 \def\at@end@of@kframe{}%
 \ifinner\ifhmode%
  \def\at@end@of@kframe{\end{minipage}}%
  \begin{minipage}{\columnwidth}%
 \fi\fi%
 \def\FrameCommand##1{\hskip\@totalleftmargin \hskip-\fboxsep
 \colorbox{shadecolor}{##1}\hskip-\fboxsep
     % There is no \\@totalrightmargin, so:
     \hskip-\linewidth \hskip-\@totalleftmargin \hskip\columnwidth}%
 \MakeFramed {\advance\hsize-\width
   \@totalleftmargin\z@ \linewidth\hsize
   \@setminipage}}%
 {\par\unskip\endMakeFramed%
 \at@end@of@kframe}
\makeatother

\definecolor{shadecolor}{rgb}{.97, .97, .97}
\definecolor{messagecolor}{rgb}{0, 0, 0}
\definecolor{warningcolor}{rgb}{1, 0, 1}
\definecolor{errorcolor}{rgb}{1, 0, 0}
\newenvironment{knitrout}{}{} % an empty environment to be redefined in TeX

\usepackage{alltt}
% --- --- --- --- --- --- --- --- --- --- --- --- --- --- --- --- --- --- --- --- 
% --- --- --- --- --- --- --- --- --- --- --- --- --- --- --- --- --- --- --- --- 
% --- --- --- --- --- --- --- --- --- --- --- --- --- --- --- --- --- --- --- --- 
\usepackage[T1]{fontenc}
\usepackage[UTF8]{inputenc}
\usepackage{tgschola}
\usepackage{textcomp}
\usepackage{tabularx}
\usepackage{indentfirst}
\usepackage{parskip}
\usepackage{array}
\usepackage{siunitx}
\usepackage{float}
% --- --- --- --- --- --- --- --- --- --- --- --- --- --- --- --- --- --- --- --- 
% --- --- --- --- --- --- --- --- --- --- --- --- --- --- --- --- --- --- --- --- 
% --- --- --- --- --- --- --- --- --- --- --- --- --- --- --- --- --- --- --- --- 
\setlength{\parindent}{6.5ex}  
\setlength{\parskip}{1.5em} 
\sisetup{
    detect-mode=false,
    mode=text, 
    round-mode = places, 
    round-precision = 2, 
    output-decimal-marker={.}, 
    group-separator ={,}, 
    group-minimum-digits = 3
}
% --- --- --- --- --- --- --- --- --- --- --- --- --- --- --- --- --- --- --- --- 
% --- --- --- --- --- --- --- --- --- --- --- --- --- --- --- --- --- --- --- --- 
% --- --- --- --- --- --- --- --- --- --- --- --- --- --- --- --- --- --- --- --- 

% --- --- --- --- --- --- --- --- --- --- --- --- --- --- --- --- --- --- --- --- 
% --- --- --- --- --- --- --- --- --- --- --- --- --- --- --- --- --- --- --- --- 
% --- --- --- --- --- --- --- --- --- --- --- --- --- --- --- --- --- --- --- --- 
% https://tex.stackexchange.com/questions/297892/old-style-financial-statement
% --- --- --- --- --- --- --- --- --- --- --- --- --- --- --- --- --- --- --- --- 
\newcolumntype{I}{@{}X<{\xdotfill}}
\newcolumntype{R}{>{\qq}r<{\qq}}
\makeatletter
\newcommand{\thickhline}{%
    \noalign {\ifnum 0=`}\fi \hrule height 1pt
    \futurelet \reserved@a \@xhline
}
\newcommand{\xdotfill}{\leavevmode\leaders\hb@xt@.44em{\hss.\hss}\hfill\hskip-\tabcolsep\kern\z@}
\makeatother
\newcommand\qq{\quad}
\newcommand\tablesec[1]{\multicolumn{1}{@{}l}{#1}&&\\}
\newcommand\tabletitle[1]{\multicolumn{1}{@{}c|}{\SEPx{2}#1}}
\newcommand\SEPx[1]{\vrule width 0pt height \dimexpr\fontcharht\font`A+2ex depth #1ex\relax}
\newcommand\SEP{\SEPx{0}}
\newcommand\?{\hphantom{0}}
\renewcommand{\arraystretch}{1.2}
% --- --- --- --- --- --- --- --- --- --- --- --- --- --- --- --- --- --- --- --- 
\tolerance=1
\emergencystretch=\maxdimen
\hyphenpenalty=10000
\hbadness=10000
% --- --- --- --- --- --- --- --- --- --- --- --- --- --- --- --- --- --- --- --- 
\title{Study Notes\\Financial Accounting: An Introduction to Concepts, Methods, and Uses, 14e\\
by Roman Weil, Katherine Schipper, Jennifer Francis}% --- --- --- --- --- --- --- --- --- --- --- --- ---
\author{Emerson Maurício de Oliveira}% --- --- --- --- --- --- --- --- --- --- --
% --- --- --- --- --- --- --- --- --- --- --- --- --- --- --- --- --- --- --- ---
\IfFileExists{upquote.sty}{\usepackage{upquote}}{}
\begin{document}
\maketitle

\section{Problem 1.2 for Self-Study}

\textbf{Cash versus accrual basis of accounting}. \textit{Thompson Hardware Store} commences 
operations on January 1, 2013, when Jacob Thompson invests \$\num{30000} for all of the common 
stock of the firm. The firm rents a building on January 1 and pays two months' rent in advance 
in the amount of \$\num{2000}. On January 1 it also pays the \$\num{1200} premium for property 
and liability insurance coverage for the year ending December 31, 2013. The firm purchases 
\$\num{28000} of merchandise inventory on account on January 2 and pays \$\num{10000} of this 
amount on January 25. On January 31 the cost of unsold merchandise is \$\num{15000}. During 
January the firm makes cash sales to customers totaling \$\num{20000} and sales on account 
totaling \$\num{9000}. The firm collects \$\num{2000} from these credit sales by the end of 
January. The firm pays other costs during January as follows: utilities, \$\num{400}; salaries, 
\$\num{650}; and taxes, \$\num{350}. What are \textit{Thompson Hardware Store's} revenues, 
expenses, and income for January, assuming (1) the \textit{accrual basis of accounting} and 
(2) the \textit{cash basis of accounting}?

\newpage

Cash Basis of accounting:\par

\begin{knitrout}\scriptsize
\definecolor{shadecolor}{rgb}{0.969, 0.969, 0.969}\color{fgcolor}\begin{kframe}
\begin{alltt}
\hlstd{creditSales} \hlkwb{<-} \hlnum{2000.00}
\hlstd{totCashInflo} \hlkwb{<-} \hlstd{creditSales}
\hlstd{cashPaidRent} \hlkwb{<-} \hlopt{-}\hlnum{2000.00}
\hlstd{cashPaidEnsu} \hlkwb{<-} \hlopt{-}\hlnum{1200.00}
\hlstd{cashPaidMerc} \hlkwb{<-} \hlopt{-}\hlnum{10000.00} \hlopt{-} \hlnum{5000.00}
\hlstd{cashPaidUtil} \hlkwb{<-} \hlopt{-}\hlnum{400.00}
\hlstd{cashPaidSala} \hlkwb{<-} \hlopt{-}\hlnum{650.00}
\hlstd{cashPaidTaxe} \hlkwb{<-} \hlopt{-}\hlnum{350.00}
\hlstd{totCashOutfl} \hlkwb{<-} \hlstd{cashPaidRent} \hlopt{+} \hlstd{cashPaidEnsu} \hlopt{+} \hlstd{cashPaidMerc} \hlopt{+} \hlstd{cashPaidUtil} \hlopt{+} \hlstd{cashPaidSala} \hlopt{+} \hlstd{cashPaidTaxe}
\hlstd{netCashFlow}  \hlkwb{<-} \hlstd{totCashInflo} \hlopt{+} \hlstd{totCashOutfl}
\end{alltt}
\end{kframe}
\end{knitrout}

% --- --- --- --- --- --- --- --- --- --- --- --- --- --- --- --- --- --- --- --- 
\begin{table}[H]
\begin{tabularx}{\linewidth}{I R R}
\thickhline
\tablesec{\textbf{Cash Inflows}}
Cash Receipts from Customers                & \texteuro\hphantom{(}\num{2000}\hphantom{)}\\
Total Cash Inflows                          & \hphantom{(}\num{2000}\hphantom{)}\\
\tablesec{\textbf{Cash Outflows}}
Cash Paid for Rent                          & (\num{2000})\\
Cash Paid for Ensurance                     & (\num{1200})\\
Cash Paid for Merchandise                   & (\num{15000})\\
Cash Paid for Utilities                     & (\num{400})\\
Cash Paid for Salaries                      & (\num{650})\\
Cash Paid for Taxes                         & (\num{350})\\
Total Cash Outflows                         & (\num{19600})\\
Net Cash Flow                               & \texteuro\hphantom{(}(\num{17600})\\
\thickhline
\end{tabularx}
\end{table}
% --- --- --- --- --- --- --- --- --- --- --- --- --- --- --- --- --- --- --- --- 

Accrual basis of accounting:\par

\begin{knitrout}\scriptsize
\definecolor{shadecolor}{rgb}{0.969, 0.969, 0.969}\color{fgcolor}\begin{kframe}
\begin{alltt}
\hlstd{salesRevenu} \hlkwb{<-} \hlnum{20000.00}
\hlstd{costGoodsSo} \hlkwb{<-} \hlopt{-}\hlnum{13000.00}
\hlstd{rentExpense} \hlkwb{<-} \hlopt{-}\hlnum{100.00}
\hlstd{salariesExp} \hlkwb{<-} \hlopt{-}\hlnum{450.00}
\hlstd{totExpenses} \hlkwb{<-} \hlstd{costGoodsSo} \hlopt{+} \hlstd{rentExpense} \hlopt{+} \hlstd{salariesExp}
\hlstd{netIncome} \hlkwb{<-} \hlstd{salesRevenu} \hlopt{+} \hlstd{totExpenses}
\end{alltt}
\end{kframe}
\end{knitrout}

% --- --- --- --- --- --- --- --- --- --- --- --- --- --- --- --- --- --- --- --- 
\begin{table}[H]
\begin{tabularx}{\linewidth}{I R R}
\thickhline\SEP
Sales Revenue                               & \$\hphantom{(}\num{20000}\hphantom{)}\\
Cost of Goods Sold                          & (\num{13000})\\
Rent Expense                                & (\num{100})\\
Salaries Expense                            & (\num{450})\\
Net Income                                  & \$\hphantom{(}\num{6450}\hphantom{)}\\[2ex]
\thickhline 
\end{tabularx}
\end{table}
% --- --- --- --- --- --- --- --- --- --- --- --- --- --- --- --- --- --- --- --- 

\section{Summary}

Cash basis of accounting:\par

Income:   \num{2000}\par
Expenses: \num{-19600}\par
Revenues: \num{-17600}\par

Accrual basis of accounting:\par

Income:   \num{20000}\par
Expenses: \num{-13550}\par
Revenues: \num{6450}\par


% --- --- --- --- --- --- --- --- --- --- --- --- --- --- --- --- --- --- --- --- 
% --- --- --- --- --- --- --- --- --- --- --- --- --- --- --- --- --- --- --- --- 
% --- --- --- --- --- --- --- --- --- --- --- --- --- --- --- --- --- --- --- --- 
% \renewcommand{\arraystretch}{1.2}
% \noindent
% \begin{tabularx}{\linewidth}{I|R|R}
% \thickhline
% \tabletitle{Item}                         & \multicolumn{1}{c|}{1939}           & \multicolumn{1}{c}{1940}\\
% \hline\SEP
% Sales                                     & \$187,400                           & \$468,300 \\
% Net income                                & 18,284                              & 27,684 \\
% \tablesec{Dec. 31 figures:}
% \qq Inventory                             & 44,163                              & 74,452 \\
% \qq Total current assets                  & 76,995                              & 109,481\\
% \qq Current ratio                         & 3.7:1\?                             & 2.0:1\? \\
% \qq Working capital per dollar of sales   & 41\textcent\?                       & 23\textcent\? \\
% \tablesec{Per share of common:}
% \qq Earned in year                        & \$4.22\?\?                          & \$10.02\?\? \\
% \qq Dividend                              & 4.00\?\?                            & 4.00\?\? \\
% \qq Net-asset value                       & 45\?                                & 76\? \\[2ex]
% \thickhline 
% \end{tabularx}
% --- --- --- --- --- --- --- --- --- --- --- --- --- --- --- --- --- --- --- --- 
% --- --- --- --- --- --- --- --- --- --- --- --- --- --- --- --- --- --- --- --- 
% --- --- --- --- --- --- --- --- --- --- --- --- --- --- --- --- --- --- --- --- 
\end{document}
